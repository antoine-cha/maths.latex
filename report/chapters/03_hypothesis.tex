
%! TEX root = ../main.tex
\documentclass[../main.tex]{subfiles}
\begin{document}

% ============================================================ %
% ============================================================ %
\chapter{Hypothesis testing}
% ============================================================ %
% ============================================================ %
\section{Confidence interval}
A confidence interval $CI_{1-\alpha}$ with confidence level $1-\alpha$ of a true parameter $\theta$
such that
\begin{align*}
    P(\theta \in CI_{1-\alpha}) = 1 - \alpha
\end{align*}

\section{Hypothesis testing}
Let $T$ be the test statistics and $R$ the rejection region.
A test statistic is a statistic (a quantity derived from the sample) used in statistical hypothesis testing.
A hypothesis test is typically specified in terms of a test statistic, considered as a numerical summary
of a dataset that reduces the data to one value that can be used to perform the hypothesis test

\subsection{Type I error}
The type I error, often noted $\alpha$, also called "false alarm" or significance level is the probability
of rejecting the null hypothesis when the null hypothesis is true.
\begin{align*}
    \alpha = P(T \in R\; |\; H_0\; true)
\end{align*}

\subsection{Type II error}
The type II error, often noted $\beta$ also called "missed alarm" is the probability
of not rejecting the null hypothesis when the null hypothesis is false.
\begin{align*}
    \alpha = P(T \notin R\; |\; H_0\; false)
\end{align*}

\subsection{p-value}
The $p$-value is the probability under the null hypothesis of a having a test statistic $T$ at
least as extreme as the one that we observed $T_0$:
\begin{align*}
    \text{(left-sided)} \quad & \text{p-value} = P(T \leq T_0\; |\; H_0\; true) \\
    \text{(right-sided)}\quad & \text{p-value} = P(T \geq T_0\; |\; H_0\; true) \\
    \text{(two-sided)}  \quad & \text{p-value} = P(|T| \geq |T_0|\; |\; H_0\; true) \\
\end{align*}

\subsection{Non-parametric test}
A non-parametric test is a test where we do not have any underlying assumption regarding the
distribution of the sample.


\end{document}
