%! TEX root = ../main.tex
\documentclass[../main.tex]{subfiles}

\begin{document}
% ============================================================ %
% ============================================================ %
\chapter{Multiple view geometry}
This chapter is heavily inspired from \cite{multi-view-geo}.

% ============================================================ %
% ============================================================ %
\section{Introduction}
\subsection{Definitions}
A point in Euclidean 2-space is represented by $(x, y)$ we may add an extra coordinate to obtain the
triple $(x,y,1)$, that we declare to represent the same point. \\
For any non-zero $k$, $(kx, ky, k)$ represents the same point as $(x,y,1)$. These are called \textbf{the
homogeneous coordinates of the point}. Formally, points are represented by equivalence classes of
coordinates triples.\\
$(x, y, 0)$ represents the points at infinity. \\
A linear transformation of Euclidean space $\mathbb{R}^n$ is represented by a $n \times n$ matrix
multiplication applied to the coordinates of the point. \\
A projective transformation of projective space $\mathbb{P}^n$ is represented by a $(n+1)\times(n+1)$
matrix multuplication applied to the homogeneous coordinates. \\
In Euclidean (or affine) geometry, infinity points remain at infinity after a linear tranform. In projective
geometry, infinity may not be preserved. \\
\\
In computer vision, the projective space is used as a convenient way of representing the world:
3D projective space for the real 3D world, and 2D projective space for 2D points in images. \\

\paragraph{Projective space}
A projective space may be viewed as the extension of a Euclidean space with points at $\infty$. \\
In a real case, a point at $\infty$ completes a line in a topologically closed curve. In igher dimensions,
all the points at $\infty$ form a projective subspace of $n-1$ dimensions. \\
\textbf{Example:} in 2D,  the ensemble of points at $\infty$  have homogeneous coordinates like
$(x, y, 0)$.  These projective points are each uniquely defined by $\dfrac{x}{y}$, where $y=0$ gives the
limit $(x, 0, 0)$ and $y=\infty$ gives the points $(0, y, 0)$. There exists a bijection from $\mathbb{R^+}$ to
the set of $\infty$ points in 2D. So this set is 1D.
\paragraph{Computer vision}
In computer vision, the origin will be the camera. Thus, the homogenous points are the lines going from the camera
to the X and beyond. For the screen at various distance, we consider various $w$ in $(wx, wy, w)$.


\subsection{Affine and euclidean geometry}
\begin{displayquote}
    \enquote{The points at infinity in the world plane correspond to a real horizon line in the image,
        and parallel lines in the world correspond to lines meeting at the horizon.}
\end{displayquote}





\section{Single and two-view geometry}
\subsection{Single view gemometry}
\paragraph{Perspective projection}
The Z-axis is the optical axis. So when, when $z$ increases, the image is further away from the camera.

\begin{equation}
    \begin{pmatrix} x \\ y \\ f \end{pmatrix}
    =
    \begin{bmatrix}
        1 & 0 & 0 & 0 \\
        0 & 1 & 0 & 0 \\
        0 & 0 & 1 & 0 \\
    \end{bmatrix}
    \begin{pmatrix} X \\ Y \\ Z \\ 1 \end{pmatrix}
\end{equation}
where a $3\times4$ \textbf{projection matrix} represents a map from 3D to 2D. $f$ the scale factor
defines the distance on the z-axis from the camera. \\
The left hand part is in \textbf{the image space}, the right hand in the \textbf{real world}.

\paragraph{Internal camera parameters}
\begin{equation}
    \textbf{$\mathrm{x}$} =
    \begin{pmatrix} x \\ y \\ 1 \end{pmatrix}
    = \frac{1}{f}
    \begin{bmatrix}
        \alpha_x &  & x_0 \\
                 & \alpha_y & y_0 \\
                 &  & 1 \\
    \end{bmatrix}
    \begin{pmatrix} x_{cam} \\ y_{cam} \\ f \end{pmatrix}
\end{equation}



\subsection{Details}
A perspective (central) projection cameraHartley04c is represented by a 3x4 matrix. \\
The most general perspective transformation between 2 planes (a world plane and the image plane, or
2 planes induced by a world plane) is a \textbf{plane projective transformation}.This can be computed from
the correspondence of 4+ points. \\
The epipolar geometry between 2 views is represented by \textbf{the fundamental matrix}. This can be computed
from 7+ points.


\section{RANdom SAmple Consensus (RANSAC)}


\end{document}
