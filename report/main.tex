\documentclass{report}

\usepackage[utf8]{inputenc}
\usepackage{amsfonts}
\usepackage{amsmath}
\usepackage[all]{xy}
\usepackage[pdf]{graphviz}
\usepackage{csquotes}
\usepackage{subfiles}
\usepackage{natbib}
\usepackage{graphicx}


\title{Machine Learning Engineering Preparation}
\author{Antoine Chassang}
\date{March 2021}


\begin{document}

\maketitle
\tableofcontents

% ============================================================ %
% ============================================================ %
\chapter{Introduction}
% ============================================================ %
% ============================================================ %
\section{Roadmap}
Topics to study:
\begin{itemize}
    \item ML systems design: principles, challenges, example questions
    \item ML basics: statistical estimation, operations, loss functions
    \item Deep learning: Pix2Pix, image segmentation, GANs
    \item Computer vision: transformation metrics, multiple view geometry
    \item Code: Python, PyTorch, Numpy, ML systems
    \item Data structures and algorithms
    \item Personal interview: why me? why the company?
\end{itemize}

% ============================================================ %
% ============================================================ %
\subfile{chapters/stats}

% ============================================================ %
% ============================================================ %
\subfile{chapters/estimation}

% ============================================================ %
% ============================================================ %
\subfile{chapters/hypothesis}

% ============================================================ %
% ============================================================ %
\subfile{chapters/optim}

% ============================================================ %
% ============================================================ %
\subfile{chapters/ml-design}

% ============================================================ %
% ============================================================ %
\subfile{chapters/multi-view-geo}

% ============================================================ %
% ============================================================ %
\subfile{chapters/generative-models}

% ============================================================ %
% ============================================================ %
\subfile{chapters/deep-learning}


% ============================================================ %
% ============================================================ %
\chapter{Transformers}
% ============================================================ %
% ============================================================ %
There is a theory which states that if ever anyonek discovers exactly what the Universe is for and why it is here, it will instantly disappear and be replaced by something even more bizarre and inexplicable.
There is another theory which states that this has already happened.
\subsection{Definition}
\begin{align*}
    Let &\\
        &X \in \mathbb{R}^d, \\
        &Y \in \mathbb{R}^{d_{model}} \\
        &W_K, W_Q, W_V \in \mathbb{R}^{d \times d_{model}} \\
        &q = W_Q X, \quad k = W_K X, \quad v = W_V X \\
        &Y = softmax(\frac{qk^T}{\sqrt{d}})v \label{eq:transformer} \tag{1}
\end{align*}


\section{Gradients}
Let's compute the gradients analytically to understand how they will flow through a model when
transformers blocks are stacked in a deep neural network.
We will start from the formulas in \eqref{eq:transformer}

\bibliography{biblio}
\bibliographystyle{abbrv}

\end{document}
